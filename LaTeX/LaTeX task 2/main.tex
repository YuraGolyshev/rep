\documentclass{article}
\usepackage[T2A]{fontenc}
\usepackage[utf8]{inputenc}
\usepackage[russian]{babel}

\begin{document}
\begin{center}\section*{\textbf{\hugeДАНИИЛ ХАРМС}}

\subsection*{\textbf{\largeОХОТНИКИ}}\end{center}
\small

На охоту поехало шесть человек, а вернулось"=то только четыре.

Двое"=то не вернулось.

Окнов, Козлов, Стрючков и Мотыльков благополучно вернулись домой, а Широков и Каблуков погибли на охоте.

Окнов целый день  ходил потом расстроенный и даже не хотел ни с кем  разговаривать.

Козлов неотступно ходил следом за Окновым и приставал к нему с различными вопросами, чем и довел Окнова до высшей точки раздражения.

КОЗЛОВ: Хочешь закурить?

ОКНОВ: Нет.

КОЗЛОВ: Хочешь, я тебе принесу вон ту вон штуку?

ОКНОВ: Нет.

КОЗЛОВ:  Может быть, хочешь, я тебе расскажу что"=нибудь смешное?

ОКНОВ: Нет.
    
КОЗЛОВ: Ну, хочешь пить? У меня вот тут есть чай с коньяком.
    
ОКНОВ: Мало того, что я тебя сейчас этим камнем по затылку ударил, я тебе еще оторву ногу.

СТРЮЧКОВ  и  МОТЫЛЬКОВ: Что вы делаете? Что вы делаете?
    
КОЗЛОВ: Приподнимите меня с земли.
    
МОТЫЛЬКОВ: Ты не волнуйся, рана заживет.
    
КОЗЛОВ: А где Окнов?
    
ОКНОВ (отрывая Козлову ногу): Я тут, недалеко!
    
КОЗЛОВ: Ох, матушки! Спа"=па"=си!
    
СТРЮЧКОВ и МОТЫЛЬКОВ: Никак он ему и ногу оторвал!
    
ОКНОВ: Оторвал и бросил вон туда!
    
СТРЮЧКОВ: Это злодейство!
    
ОКНОВ: Что"=о?
    
СТРЮЧКОВ: \dots ейство\dots
    
ОКНОВ: Ка"=а"=ак?
    
СТРЮЧКОВ: Нь\dotsнь\dotsнь\dotsникак.
    
КОЗЛОВ: Как же я дойду до дому?
    
МОТЫЛЬКОВ: Не беспокойся, мы тебе приделаем деревяшку.
    
СТРЮЧКОВ: Ты на одной ноге стоять можешь?
    
КОЗЛОВ: Могу, но не очень"=то.
    
СТРЮЧКОВ: Ну, мы тебя поддержим.
    
ОКНОВ: Пустите меня к нему!
    
СТРЮЧКОВ: Ой нет, лучше уходи!
    
ОКНОВ: Нет, пустите!.. Пустите!.. Пусти\dots Вот, что я хотел сделать.
    
СТРЮЧКОВ и МОТЫЛЬКОВ: Какой ужас!
    
ОКНОВ: Ха"=хa"=ха!
    
МОТЫЛЬКОВ: А где же Козлов?
    
СТРЮЧКОВ: Он уполз в кусты.
    
МОТЫЛЬКОВ: Козлов, ты тут?
    
КОЗЛОВ: Шаша!..
    
МОТЫЛЬКОВ: Вот ведь до чего дошел!
    
СТРЮЧКОВ: Что же с ним делать?
    
МОТЫЛЬКОВ: А тут уж ничего с ним не поделаешь. По"=моему, его надо просто удавить. Козлов! А, Козлов? Ты меня слышишь?
    
КОЗЛОВ: Ох, слышу, да плохо.
    
МОТЫЛЬКОВ: Ты, брат, не горюй. Мы сейчас тебя удавим. Постой!.. Вот\dots Вот\dots Вот\dots

СТРЮЧКОВ: Вот сюда вот еще! Так! Так! Так! Ну"=ка еще\dots Ну, теперь готово!
    
МОТЫЛЬКОВ: Теперь готово!
    
ОКНОВ: Господи, благослови!

\begin{flushright}<\dots>\end{flushright}
\begin{center}

***

\subsection*{\textbf{\largeИВАН ТОПОРЫШКИН}}\end{center}
\begin{quote}
Иван Топорышкин пошёл на охоту, 

С ним пудель пошёл, перепрыгнув забор. 

Иван, как бревно, провалился в болото, 

А пудель в реке утонул, как топор. 

\begin{verse}
Иван Топорышкин пошёл на охоту, 

С ним пудель вприпрыжку пошёл, как топор. 

Иван повалился бревном на болото, 

А пудель в реке перепрыгнул забор.

\begin{verse}
Иван Топорышкин пошёл на охоту, 

С ним пудель в реке провалился в забор. 

Иван, как бревно, перепрыгнул болото, 

А пудель вприпрыжку попал на топор. 
\end{verse}
\end{verse}
\end{quote}
\begin{flushright}1928 г.\end{flushright}

\begin{center}

***

\subsection*{\textbf{\largeИСТОРИЧЕСКИЙ ЭПИЗОД}}\end{center}

\begin{flushright}В.Н.Петрову\end{flushright}

Иван Иванович Сусанин (то самое историческое лицо, которое положило свою жизнь  за царя и впоследствии было воспето оперой Глинки) зашел однажды в русскую харчевню и, сев за стол, потребовал себе антрекот. Пока хозяин харчевни жарил антрекот, Иван Иванович закусил свою бороду зубами и задумался: такая у него была привычка.
    
Прошло 35 колов времени, и хозяин принес Ивану Ивановичу антрекот на круглой деревянной дощечке. Иван Иванович был голоден и по обычаю того времени схватил антрекот руками и начал его есть. Но, торопясь утолить свой голод, Иван Иванович так жадно набросился на антрекот, что забыл вынуть изо рта свою бороду и съел антрекот с куском своей бороды.

Вот тут"=то и произошла неприятность, так как не прошло и пятнадцатим колов времени, как в животе у Ивана Ивановича начались сильные рези. Иван Иванович вскочил из"=за стола и кинулся на двор. Хозяин крикнул было Ивану Ивановичу: \textit{<<Зри, како твоя борода клочна>>}, "--* но Иван Иванович, не обращая ни на что внимания, выбежал во двор.
    
Тогда боярин Ковшегуб,  сидящий  в  углу харчевни и пьющий сусло, ударил  кулаком  по столу и вскричал: \textit{<<Кто есть сей?>>}  А хозяин, низко кланяясь, ответил боярину: \textit{<<Сие есть наш патриот  Иван  Иванович  Сусанин>>}."--* \textit{<<Во как!>>} "--* сказал боярин, допивая свое сусло.
    
\textit{<<Не  угодно ли рыбки?>>} "--* спросил хозяин. \textit{<<Пошел ты к бую!>>} "--* крикнул боярин и  пустил в  хозяина ковшом. Ковш просвистел возле хозяйской головы, вылетел через окно на двор и хватил по зубам  сидящего орлом Ивана Ивановича. Иван Иванович  схватился рукой за щеку и повалился на бок.
    
Тут справа из сарая выбежал  Карп и, перепрыгнув через корыто, на котором среди помоев лежала свинья, с криком побежал к воротам. Из харчевни выглянул хозяин. \textit{<<Чего ты орешь?>>} "--* спросил он Карпа. Но Карп,  ничего не отвечая, убежал.
    
Хозяин вышел на двор и увидел  Сусанина, лежащего неподвижно на земле. Хозяин подошел поближе  и заглянул ему в лицо. Сусанин пристально глядел на хозяина.  \textit{<<Так ты жив?>>} "--* спросил  хозяин. \textit{<<Жив, да тилько страшусь, что  меня еще чем"=нибудь ударят>>}, "--*  сказал Сусанин. \textit{<<Нет,"--* сказал хозяин,"--*  не  страшись. Это тебя боярин Ковшегуб чуть не убил, а теперь он ушедши>>}. \textit{<<Ну и слава тебе, Боже}, "--* сказал Иван Сусанин, поднимаясь с земли."--* \textit{Я человек храбрый, да тилько  зря  живот покладать  не  люблю.  Вот и приник к земле  и ждал, что дальше будет. Чуть чего,  я бы  на животе до самой Елдыриной слободы бы  уполз. Евона как щеку разнесло. Батюшки!  Полбороды отхватило!>>}.. \textit{<<Это у тебя еще раньше было>>}, "--* сказал хозяин. \textit{<<Как это так раньше?} "--* вскричал патриот Сусанин. "--* \textit{Что же,  по"=твоему, я так с клочной бородой ходил?>>} \textit{<<Ходил>>}, "--* сказал хозяин. \textit{<<Ах ты, мяфа>>}, "--* проговорил Иван Сусанин. Хозяин зажмурил глаза и, размахнувшись  со  всего  маху, звезданул Сусанина по уху. Патриот Сусанин рухнул  на  землю и замер. \textit{<<Вот тебе! Сам ты мяфа!>>} "--* сказал хозяин и удалился в харчевню.
    
Несколько колов времени Сусанин лежал на земле и прислушивался, но, не слыша ничего подозрительного, осторожно приподнял голову и осмотрелся. На дворе никого не  было, если не считать  свиньи, которая вывалившись из корыта, валялась теперь в грязной луже. Иван Сусанин, озираясь, подобрался к воротам. Ворота, по счастью, были открыты, и патриот Иван Сусанин, извиваясь по земле, как червь, пополз по направлению к Елдыриной слободе.
    
Вот эпизод из жизни знаменитого исторического лица, которое положило свою жизнь за царя и было впоследствии воспето в опере Глинки.

\begin{flushright}<\dots>\end{flushright}

\end{document}

