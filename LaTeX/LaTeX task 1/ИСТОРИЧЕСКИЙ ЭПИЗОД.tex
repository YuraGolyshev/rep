\begin{flushright}В.Н.Петрову\end{flushright}                                
Иван Иванович Сусанин (то самое историческое лицо, которое положило свою жизнь  за царя и впоследствии было воспето оперой Глинки) зашел однажды в русскую харчевню и, сев за стол, потребовал себе антрекот. Пока хозяин харчевни жарил антрекот, Иван Иванович закусил свою бороду зубами и задумался: такая у него была привычка.
    
Прошло 35 колов времени, и хозяин принес Ивану Ивановичу антрекот на круглой деревянной дощечке. Иван Иванович был голоден и по обычаю того времени схватил антрекот руками и начал его есть. Но, торопясь утолить свой голод, Иван Иванович так жадно набросился на антрекот, что забыл вынуть изо рта свою бороду и съел антрекот с куском своей бороды.

Вот тут"=то и произошла неприятность, так как не прошло и пятнадцатим колов времени, как в животе у Ивана Ивановича начались сильные рези. Иван Иванович вскочил из"=за стола и кинулся на двор. Хозяин крикнул было Ивану Ивановичу: <<Зри, како твоя борода клочна>>,"--* но Иван Иванович, не обращая ни на что внимания, выбежал во двор.
    
Тогда боярин Ковшегуб,  сидящий  в  углу харчевни и пьющий сусло, ударил  кулаком  по столу и вскричал: <<Кто есть сей?>>  А хозяин, низко кланяясь, ответил боярину: <<Сие есть наш патриот  Иван  Иванович  Сусанин>>."--* <<Во как!>>"--* сказал боярин, допивая свое сусло.
    
<<Не  угодно ли рыбки?>>"--* спросил хозяин. <<Пошел ты к бую!>>"--* крикнул боярин и  пустил в  хозяина ковшом. Ковш просвистел возле хозяйской головы, вылетел через окно на двор и хватил по зубам  сидящего орлом Ивана Ивановича. Иван Иванович  схватился рукой за щеку и повалился на бок.
    
Тут справа из сарая выбежал  Карп и, перепрыгнув через корыто, на котором среди помоев лежала свинья, с криком побежал к воротам. Из харчевни выглянул хозяин. <<Чего ты орешь?>>"--* спросил он Карпа. Но Карп,  ничего не отвечая, убежал.
    
Хозяин вышел на двор и увидел  Сусанина, лежащего неподвижно на земле. Хозяин подошел поближе  и заглянул ему в лицо. Сусанин пристально глядел на хозяина.  <<Так ты жив?>>"--* спросил  хозяин. <<Жив, да тилько страшусь, что  меня еще чем"=нибудь ударят>>,"--*  сказал Сусанин. <<Нет,"--* сказал хозяин,"--*  не  страшись. Это тебя боярин Ковшегуб чуть не убил, а теперь он ушедши>>. <<Ну и слава тебе, Боже, "--* сказал Иван Сусанин, поднимаясь с земли."--* Я человек храбрый, да тилько  зря  живот покладать  не  люблю.  Вот и приник к земле  и ждал, что дальше будет. Чуть чего,  я бы  на животе до самой Елдыриной слободы бы  уполз. Евона как щеку разнесло. Батюшки!  Полбороды отхватило!>>.. <<Это у тебя еще раньше было>>,"--* сказал хозяин. <<Как это так раньше?"--* вскричал патриот Сусанин."--* Что же,  по"=твоему, я так с клочной бородой ходил?>> <<Ходил>>,"--* сказал хозяин. <Ах ты, мяфа", "--* проговорил Иван Сусанин. Хозяин зажмурил глаза и, размахнувшись  со  всего  маху, звезданул Сусанина по уху. Патриот Сусанин рухнул  на  землю и замер. <<Вот тебе! Сам ты мяфа!>>"--* сказал хозяин и удалился в харчевню.
    
Несколько колов времени Сусанин лежал на земле и прислушивался, но, не слыша ничего подозрительного, осторожно приподнял голову и осмотрелся. На дворе никого не  было, если не считать  свиньи, которая вывалившись из корыта, валялась теперь в грязной луже. Иван Сусанин, озираясь, подобрался к воротам. Ворота, по счастью, были открыты, и патриот Иван Сусанин, извиваясь по земле, как червь, пополз по направлению к Елдыриной слободе.
    
Вот эпизод из жизни знаменитого исторического лица, которое положило свою жизнь за царя и было впоследствии воспето в опере Глинки.
