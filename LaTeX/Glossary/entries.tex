\begin{flushleft} \large\textbf{A} \end{flushleft}

\begin{enumerate}
    \setcounter{enumi}{0}

    \item \textbf{Agile} (\textit{Александр Кузьмин}) "--- 
    это итеративный подход к управлению проектами и разработке программного обеспечения, который помогает командам быстрее и с меньшими проблемами поставлять ценность клиентам. Вместо того чтобы выпускать весь продукт целиком, команда, следующая принципам Agile, выполняет работу в рамках небольших, но удобных инкрементов. Требования, планы и результаты оцениваются непрерывно, благодаря чему команды могут быстро реагировать на изменения.
    
\end{enumerate}

\begin{flushleft} \large\textbf{B} \end{flushleft}

\begin{enumerate}
    \setcounter{enumi}{1}

    \item \textbf{Big data} (\textit{Александр Кузьмин}) "--- 
    обозначение структурированных и неструктурированных данных огромных объёмов и значительного многообразия, эффективно обрабатываемых горизонтально масштабируемыми программными инструментами, появившимися в конце 2000"=х годов и альтернативных традиционным системам управления базами данных и решениям класса Business Intelligence.

    \item \textbf{Business intelligence} (\textit{Александр Кузьмин}) "---
     обозначение компьютерных методов и инструментов для организаций, обеспечивающих перевод транзакционной деловой информации в человекочитаемую форму, а также средства для массовой работы с такой обработанной информацией.

    \item \textbf{Blueprints} (\textit{Леонид, студент}) "---
      это визуальная, нодовая система программирования, которая используется в Unreal Engine 4. С помощью составления логических блоков нодов, можно <<собрать>> как из конструктора программу любой сложности, начиная от простого кликера, заканчивая полноценной RPG"=игрой. Так как в блупринтах не используется программный код "--- написать программу может каждый, кто понимает основные принципы ООП.
     
\end{enumerate}

\begin{flushleft} \large\textbf{D} \end{flushleft}

\begin{enumerate}
    \setcounter{enumi}{4}
    
    \item \textbf{Data science} (\textit{Александр Кузьмин}) "--- 
    наука о данных, объединяющая разные области знаний: информатику, математику и системный анализ. Сюда входят
методы обработки больших данных (Big Data), интеллектуального анализа данных (Data Mining), статистические методы, методы искусственного
интеллекта, в т. ч. машинное обучение (Machine Learning).

    \item \textbf{DevOps} (акр. от <<development \& operations>>)(\textit{Александр Кузьмин}) "--- 
    методология автоматизации технологических процессов сборки, настройки и развёртывания программного обеспечения.

    \item \textbf{Data mining} (\textit{Александр Кузьмин}) "--- 
    собирательное название, используемое для обозначения совокупности методов обнаружения в данных ранее
неизвестных, нетривиальных, практически полезных и доступных интерпретации знаний, необходимых для принятия решений в различных сферах
человеческой деятельности.

\item \textbf{Data Engineering } (\textit{Александр Кузьмин}) "--- 
    область IT, связаная с доставкой, хранением и обработкой данных. Главная задача data"=инженеров "--- обеспечить надёжную инфраструктуру для данных.
    
\end{enumerate}

\begin{flushleft} \large\textbf{F} \end{flushleft}
\begin{enumerate}
    \setcounter{enumi}{8}
    
    \item \textbf{Fast data} (\textit{Александр Кузьмин}) "--- 
    это совокупность технологий, обеспечивающих скоростную обработку потоков данных, генерируемых различными системами и активностями, их анализ и принятие решений с учетом исторически накопленных данных, а также автоматическое выполнение различных действий в ответ на произошедшие события.
    
\end{enumerate}

\begin{flushleft} \large\textbf{G} \end{flushleft}

\begin{enumerate}
    \setcounter{enumi}{9}
    
    \item \textbf{GPT} или \textbf{generative pre"=trained transformer} (рус. Генеративный предобученный трансформер) (\textit{Никита Рыданов}) "--- это тип нейронных языковых моделей, впервые представленных компанией OpenAI, которые обучаются на больших наборах текстовых данных, чтобы генерировать текст, схожий с человеческим. Предобучение относится к начальному процессу обучения на корпусе, в результате которого модель учится предсказывать следующее слово в тексте и получает основу для успешного выполнения дальнейших задач, не имея больших объёмов данных. GPT являются <<трансформерами>>, которые представляют собой тип нейросетей, использующих механизм самосвязываемости для обработки последовательных данных. Они могут быть дообучены для различных задач обработки естественного языка (NLP), таких как генерация текста, машинный перевод и классификация текста.
    
\end{enumerate}

\begin{flushleft} \large\textbf{K} \end{flushleft}

\begin{enumerate}
    \setcounter{enumi}{10}
    
    \item \textbf{Kanban} (\textit{Никита Баранов}) "--- 
    это популярный подход к реализации принципов agile и DevOps при разработке ПО. Методика предполагает обсуждение производительности в режиме реального времени и полную прозрачность рабочих процессов. Рабочие задачи визуально представлены на доске Kanban, что позволяет участникам команды видеть состояние каждой задачи в любой момент времени.
    
\end{enumerate}

\begin{flushleft} \large\textbf{M} \end{flushleft}
\begin{enumerate}
    \setcounter{enumi}{11}
    
    \item \textbf{MLOps} или \textbf{ML Ops} (\textit{Александр Кузьмин}) "--- 
    это набор практик нацеленных на надежное и эффективное развертывание и поддержание моделей машинного обучения на производстве. Слово является смесью слов "машинное обучение" (ML) и практик непрерывной разработки "--- DevOps в области программного обеспечения.

    \item \textbf{MSA}  (аббревиатура от Measurement System Analysis) (\textit{Никита Баранов}) "---
     это метод, призванный дать заключение относительно приемлемости используемой измерительной системы через количественное выражение её характеристик. Под измерительными системами понимаются совокупность приборов, стандартов, операций, методов, персонала, компьютерных программ, окружающей среды, используемых для придания количественных значений измеряемым величинам. Задачей ИС является получение данных, анализ которых применяется для принятия управленческих решений в отношении продукции или процессов.
    
\end{enumerate}

\begin{flushleft} \large\textbf{O} \end{flushleft}

\begin{enumerate}
    \setcounter{enumi}{13}
    
    \item \textbf{Open"=source} (\textit{Никита Рыданов}) "--- это программное обеспечение, распространяемое с открытым исходным кодом. Такое приложение можно доработать (изменить, дополнить) под свои задачи без нарушения авторских прав разработчиков, а также изучить на наличие уязвимостей, использовать для разработки других программ и т.д.
    
\end{enumerate}

\begin{flushleft} \large\textbf{T} \end{flushleft}

\begin{enumerate}
\setcounter{enumi}{14}
 
     \item \textbf{TCP} (\textit{Леонид, студент}) "--- 
      транспортный протокол передачи данных в сетях TCP/IP, предварительно устанавливающий соединение с сетью.
     
\end{enumerate}

\begin{flushleft} \large\textbf{U} \end{flushleft}

\begin{enumerate}
    \setcounter{enumi}{15}

    \item \textbf{UDP} (\textit{Леонид, студент}) "--- 
      транспортный протокол, передающий сообщения"=датаграммы без необходимости установки соединения в IP"=сети.
    
    \item\textbf{User Experience} (\textit{Александр Кузьмин}) "--- 
    это направление в дизайне и проектировании, в основе которого лежит пользовательский опыт. UX отвечает на вопрос, как сделать понятный и удобный продукт, чтобы человек смог легко им пользоваться. Это необязательно может быть приложение или цифровой сервис. UX "--- широкое понятие, которое можно использовать при проектировании туалета, бассейна, финансовой системы.

    \item\textbf{User Interface} (\textit{Александр Кузьмин}) "--- 
     направление в дизайне и проектировании. Если в UX продумывают концепцию продукта, изучают варианты, как сделать его удобным, простым и понятным, то в UI решают, как именно реализовать эти варианты. UX "--- про концептуальную оболочку идеи, UI "--- про графическую.
    
    
\end{enumerate}

\begin{flushleft} \large\textbf{W} \end{flushleft}

\begin{enumerate}
    \setcounter{enumi}{18}
    
\item \textbf{Web"=тестировщик} (от англ. full stack) (\textit{Павел Пасеков}) "--- 
    специалист, который оценивает функциональную работоспособность, архитектуру приложения, тестирует базы данных, выполняет нагрузочное тестирование
    
\end{enumerate}

\begin{flushleft} \large\textbf{А} \end{flushleft}

\begin{enumerate}
\setcounter{enumi}{19}
 
     \item \textbf{Алгоритм} (\textit{Леонид, студент}) "--- 
     это набор инструкций для решения конкретной проблемы или достижения конкретной задачи.

     \item \textbf{Атомарное условие} (\textit{Максим, КРЭТ}) "--- 
     это такое условие, которое нельзя более декомпозировать на более мелкие условия.
     
\end{enumerate}

\begin{flushleft} \large\textbf{Б} \end{flushleft}

\begin{enumerate}
\setcounter{enumi}{21}
\item \textbf{Бэкенд} (англ. backend) (\textit{Павел Пасеков}) "--- 
    это логика работы сайта, скрытая от пользователя. Именно там происходит то, что можно назвать работой сайта.
\end{enumerate}

\begin{flushleft} \large\textbf{В} \end{flushleft}

\begin{enumerate}
\setcounter{enumi}{22}
\item \textbf{Вирус} (\textit{Игорь Юрин}) "--- 
    вид вредоносных программ, способных внедряться в код других программ, системные области памяти, загрузочные секторы и распространять свои копии по разнообразным каналам связи.
\end{enumerate}

\begin{flushleft} \large\textbf{Г} \end{flushleft}

\begin{enumerate}
\setcounter{enumi}{23}
\item \textbf{Генеративно"=состязательная сеть} (\textit{Александр Кузьмин}) "---
     алгоритм машинного обучения без учителя, построенный на комбинации из двух нейронных сетей, одна из которых (сеть G) генерирует образцы, а другая (сеть D) старается отличить правильные (<<подлинные>>) образцы от неправильных. Так как сети G и D имеют противоположные цели "--- создать образцы и отбраковать образцы "--- между ними возникает антагонистическая игра. Генеративно"=состязательную сеть описал Ян Гудфеллоу из компании Google в 2014 году.
     
\item \textbf{Глубокое обучение} (англ. deep learning) (\textit{Александр Кузьмин}) "---
     совокупность методов машинного обучения, основанных на обучении представлениям, а не специализированных алгоритмах под конкретные задачи.

\item \textbf{Геймдев} (от англ. <<game development>>) (\textit{Леонид, студент}) "---
     процесс создания игры: от разработки и дизайна до выпуска на рынок. Это могут быть игры для мобильных телефонов, консолей, компьютеров или других гаджетов.
     

\end{enumerate}

\begin{flushleft} \large\textbf{Д} \end{flushleft}

\begin{enumerate}
\setcounter{enumi}{26}
 
     \item \textbf{Декодер}  (паттерн, от англ. design pattern) (\textit{Никита Рыданов}) "--- 
      преобразует числа в новую информацию.
     
\end{enumerate}

\begin{flushleft} \large\textbf{Ж} \end{flushleft}

\begin{enumerate}
\setcounter{enumi}{27}
\item \textbf{Железо} (\textit{Леонид, студент}) "---
      материально"=техническая часть/комплектующие компьютеров/ноутбуков/смартфонов и т.д.

\end{enumerate}

\begin{flushleft} \large\textbf{И} \end{flushleft}

\begin{enumerate}
\setcounter{enumi}{28}
\item \textbf{Информационная безопасность} (англ. Information Security, а также "--- англ. InfoSec) (\textit{Игорь Юрин}) "--- 
    практика предотвращения несанкционированного доступа, использования, раскрытия, искажения, изменения, исследования, записи или уничтожения информации. Это универсальное понятие применяется вне зависимости от формы, которую могут принимать данные (электронная или, например, физическая). Основная задача информационной безопасности "--- сбалансированная защита конфиденциальности, целостности и доступности данных, с учётом целесообразности применения и без какого"=либо ущерба производительности организации. Это достигается, в основном, посредством многоэтапного процесса управления рисками, который позволяет идентифицировать основные средства и нематериальные активы, источники угроз, уязвимости, потенциальную степень воздействия и возможности управления рисками. Этот процесс сопровождается оценкой эффективности плана по управлению рисками.
\end{enumerate}

\begin{flushleft} \large\textbf{К} \end{flushleft}

\begin{enumerate}
\setcounter{enumi}{29}
    \item \textbf{Контейнер} (\textit{Леонид, студент}) "---
     это класс STL, реализующий функциональность некоторой структуры данных, то есть хранилища нескольких элементов. Примеры разных контейнеров: vector, stack, queue, deque, string, set, map и т.д. Различные контейнеры имеют различные способы доступа к элементам.

     \item \textbf{Компьютерная безопасность} (\textit{Игорь Юрин}) "--- 
    меры безопасности, применяемые для защиты вычислительных устройств (компьютеры, смартфоны и другие), а также компьютерных сетей (частных и публичных сетей, включая Интернет). Поле деятельности системных администраторов охватывает все процессы и механизмы, с помощью которых цифровое оборудование, информационное поле и услуги защищаются от случайного или несанкционированного доступа, изменения или уничтожения данных, и приобретает всё большее значение в связи с растущей зависимостью от компьютерных систем в развитом сообществе.

\end{enumerate}

\begin{flushleft} \large\textbf{М} \end{flushleft}
\begin{enumerate}
\setcounter{enumi}{31}
    \item \textbf{Микросервисная архитектура} (\textit{Александр Кузьмин}) "---
     это подход к созданию приложения в виде набора независимо развертываемых сервисов, которые являются децентрализованными и разрабатываются независимо друг от друга. Эти сервисы слабо связаны, независимо развертываются и легко обслуживаются.

     \item \textbf{Многопоточность} (англ. Multithreading) (\textit{Леонид, студент}) "--- 
      cвойство платформы (например, операционной системы, виртуальной машины и т. д.) или приложения, состоящее в том, что процесс, порождённый в операционной системе, может состоять из нескольких потоков, выполняющихся <<параллельно>>, то есть без предписанного порядка во времени.

      \item \textbf{Машинное обучение} (англ. machine learning, ML) (\textit{Александр Кузьмин}) "---
      класс методов искусственного интеллекта, характерной чертой которых является не прямое решение задачи, а обучение за счёт применения решений множества сходных задач. Для построения таких методов используются средства математической статистики, численных методов, математического анализа, методов оптимизации, теории вероятностей, теории графов, различные техники работы с данными в цифровой форме.
\end{enumerate}

\begin{flushleft} \large\textbf{П} \end{flushleft}
\begin{enumerate}
\setcounter{enumi}{34}
 
    \item \textbf{Программисты 1С} (\textit{Иван Жадаев}) "--- 
    смпециалисты, которые занимаются внедрением и сопровождением программ 1С в организациях: устанавливают и настраивают, дорабатывают и обновляют их, а ещё консультируют пользователей.

\end{enumerate}

\begin{flushleft} \large\textbf{Р} \end{flushleft}
\begin{enumerate}

\setcounter{enumi}{35}
 
    \item \textbf{Ретроспективы agile} (\textit{Никита Баранов}) "--- 
    это любое время, когда команда размышляет о прошлом, чтобы улучшить будущее. Помимо рабочих вопросов в технических и нетехнических командах, ретроспективы можно устраивать практически по любым темам.

    \item \textbf{Рекуррентные нейронные сети} (\textit{Александр Кузьмин}) "--- 
    вид нейронных сетей, где связи между элементами образуют направленную последовательность. Благодаря этому появляется возможность обрабатывать серии событий во времени или последовательные пространственные цепочки. В отличие от многослойных перцептронов, рекуррентные сети могут использовать свою внутреннюю память для обработки последовательностей произвольной длины. Поэтому сети RNN применимы в таких задачах, где нечто целостное разбито на части, например: распознавание рукописного текста или распознавание речи.
    
\end{enumerate}

\begin{flushleft} \large\textbf{С} \end{flushleft}

\begin{enumerate}
\setcounter{enumi}{37}
 
    \item \textbf{Семантика} (\textit{Михаил Чернигин}) "--- 
    дисциплина, изучающая формализации значений конструкций языков программирования посредством построения их формальных математических моделей. В качестве инструментов построения таких моделей могут использоваться различные средства, например, математическая логика, $\lambda$"=исчисление, теория множеств, теория категорий, теория моделей, универсальная алгебра. Формализация семантики языка программирования может использоваться как для описания языка, определения свойств языка, так и для целей формальной верификации программ на этом языке программирования.

    \item \textbf{Свёрточная нейронная сеть} (англ. convolutional neural network, CNN) (\textit{Александр Кузьмин}) "--- 
    специальная архитектура искусственных нейронных сетей, предложенная Яном Лекуном в 1988 году и нацеленная на эффективное распознавание образов, входит в состав технологий глубокого обучения (англ. deep learning). Использует некоторые особенности зрительной коры, в которой были открыты так называемые простые клетки, реагирующие на прямые линии под разными углами, и сложные клетки, реакция которых связана с активацией определённого набора простых клеток. Таким образом, идея свёрточных нейронных сетей заключается в чередовании свёрточных слоёв (англ. convolution layers) и субдискретизирующих слоёв (англ. subsampling layers или англ. pooling layers, слоёв подвыборки). Структура сети "--- однонаправленная (без обратных связей), принципиально многослойная. Для обучения используются стандартные методы, чаще всего метод обратного распространения ошибки. Функция активации нейронов (передаточная функция) "--- любая, по выбору исследователя.

    \item \textbf{Спринт} (\textit{Никита Баранов}) "--- 
    это короткий временной интервал, в течение которого scrum"=команда выполняет заданный объем работы. Спринты лежат в основе методологий scrum и agile, и правильный выбор спринтов поможет вашей agile‑команде выпускать более качественное программное обеспечение без лишней головной боли.

    \item \textbf{Сервис"=ориентиированная архитектура} (СОА, англ. service"=oriented architecture"= SOA)(\textit{Никита Баранов}) "--- 
     модульный подход к разработке программного обеспечения, базирующийся на обеспечении удаленного по стандартизированным протоколам использования распределённых, слабо связанных, легко заменяемых компонентов (сервисов) со стандартизированными интерфейсами.

     \item \textbf{Структура данных} (\textit{Леонид, студент}) "--- 
     это способ организации информации для более эффективного использования. В программировании структурой обычно называют набор данных, связанных определённым образом. Например, массив "--- это структура.

     \item \textbf{Система управления базами данных} (СУБД) (\textit{Иван Жадаев}) "--- 
     это комплекс программно"=языковых средств, позволяющих создать базы данных и управлять данными. Иными словами, СУБД "--- это набор программ, позволяющий организовывать, контролировать и администрировать базы данных. Большинство сайтов не могут функционировать без базы данных, поэтому СУБД используется практически повсеместно.

     \item \textbf{Специалист технической поддержки} (СУБД) (\textit{Иван Жадаев}) "--- 
     это человек, хорошо знакомый с сервисами и проектами своего нанимателя и способный в короткие сроки решать проблемы покупателей и клиентов. 
\end{enumerate}

\begin{flushleft} \large\textbf{Т} \end{flushleft}

\begin{enumerate}
\setcounter{enumi}{44}
 
     \item \textbf{Трансформер}  (паттерн, от англ. design pattern) (\textit{Никита Рыданов}) "--- 
      архитектура глубоких нейронных сетей, представленная в 2017 году исследователями из Google Brain. По аналогии с рекуррентными нейронными сетями (РНС) трансформеры предназначены для обработки последовательностей, таких как текст на естественном языке, и решения таких задач как машинный перевод и автоматическое реферирование. В отличие от РНС, трансформеры не требуют обработки последовательностей по порядку. Например, если входные данные "--- это текст, то трансформеру не требуется обрабатывать конец текста после обработки его начала. Благодаря этому трансформеры распараллеливаются легче чем РНС и могут быть быстрее обучены.

      \item \textbf{Толстый клиент} (\textit{Иван Жадаев}) "--- 
     в архитектуре клиент "--- сервер "--- это приложение, обеспечивающее (в противовес тонкому клиенту) расширенную функциональность независимо от центрального сервера. Часто сервер в этом случае является лишь хранилищем данных, а вся работа по обработке и представлению этих данных переносится на машину клиента.

    \item \textbf{Тонкий клиент} (\textit{Иван Жадаев}) "--- 
     бездисковый компьютер"=клиент в сетях с клиент"=серверной или терминальной архитектурой, который переносит все или большую часть задач по обработке информации на сервер.
     
     \item \textbf{Троянская вирусная программа} (троян) (\textit{Игорь Юрин}) "--- 
    разновидность вредоносной программы, проникающая в компьютер под видом легитимного программного обеспечения, в отличие от вирусов и червей, которые распространяются самопроизвольно. В данную категорию входят программы, осуществляющие различные неподтверждённые пользователем действия: сбор информации о банковских картах, передача этой информации злоумышленнику, а также использование, удаление или злонамеренное изменение, нарушение работоспособности компьютера, использование ресурсов компьютера в целях майнинга, использование IP для нелегальной торговли.
     
\end{enumerate}

\begin{flushleft} \large\textbf{Ф} \end{flushleft}

\begin{enumerate}
\setcounter{enumi}{48}
    \item \textbf{Фронтенд} (англ. frontend) (\textit{Павел Пасеков}) "--- 
    это разработка пользовательских функций и интерфейса. К ним относится всё, что пользователи видят на сайте или в приложении, и с чем можно взаимодействовать: картинки, выпадающие списки, меню, анимация, карточки товаров, кнопки, чекбоксы, интерактивные элементы. На любой странице в интернете виден результат работы фронтенд"=разработчика.
    \item \textbf{Фуллстек} (от англ. full stack) (\textit{Павел Пасеков}) "--- 
    разработчик, который одновременно трудится в бэкендее и фронтенде.
\end{enumerate}

\begin{flushleft} \large\textbf{Ч} \end{flushleft}

\begin{enumerate}
\setcounter{enumi}{50}

     \item \textbf{Черный ящик} (модель) (\textit{Максим, КРЭТ}) "--- 
     это система, в которой внешнему наблюдателю доступны лишь входные и выходные величины, а структура и внутренние процессы не известны. Любая вещь, любой предмет, любое явление, любой познаваемый объект – всегда первоначально выступает как <<черный ящик>>.
     
\end{enumerate}

\begin{flushleft} \large\textbf{Ш} \end{flushleft}

\begin{enumerate}
\setcounter{enumi}{51}
 
     \item \textbf{Шаблон проектирования}  (паттерн, от англ. design pattern) (\textit{Леонид, студент}) "--- 
      повторяемая архитектурная конструкция в сфере проектирования программного обеспечения, предлагающая решение проблемы проектирования в рамках некоторого часто возникающего контекста.
     
\end{enumerate}

\begin{flushleft} \large\textbf{Э} \end{flushleft}

\begin{enumerate}
\setcounter{enumi}{52}
 
     \item \textbf{Энкодер}  (паттерн, от англ. design pattern) (\textit{Никита Рыданов}) "--- 
      обрабатывает полученную информацию и перерабатывает ее в числовой набор.
     
\end{enumerate}
