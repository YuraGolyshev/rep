\documentclass{article}
\usepackage[T2A]{fontenc}
\usepackage[utf8]{inputenc}
\usepackage{amsthm}
\usepackage{amsmath}
\usepackage{amssymb}
\usepackage{amsfonts}
\usepackage{mathrsfs}
\usepackage[12pt]{extsizes}
\usepackage{fancyvrb}
\usepackage{indentfirst}
\usepackage[
  left=2cm, right=2cm, top=2cm, bottom=2cm, headsep=0.2cm, footskip=0.6cm, bindingoffset=0cm
]{geometry}
\usepackage[english,russian]{babel}


\begin{document}
\section*{Вариант 4}
Тогда математическое ожидание оценки функции оценки перспектив будет равно
\begin{equation*}
E_0\left(u\left(\frac{\Lambda_T}{\Lambda_0}(S_T-X)\right)\right)=\int u\left(\frac{\Lambda_T}{\Lambda_0}(S_T-X)\right)df(\Lambda_T,S_T),
\end{equation*}
где $S_T$ и $\Lambda_T$ есть решение предыдущего уравнения. Имеем
\begin{equation}
\ln\Lambda_T=\ln\Lambda_0-\left(r+\frac{1}{2}\left(\frac{\mu-r}{\sigma}\right)^2\right)T-\frac{\mu-r}{\sigma}\sqrt{T}\varepsilon,
\end{equation}
где $\varepsilon\sim N(0,1)$, $S_0$ "--* начальная стоимость портфеля. Тогда, используя (1), получим
\begin{multline}
E_0\left(u\left(\frac{\Lambda_T}{\Lambda_0}(S_T-X)\right)\right)=-\int\limits_{S_T=X}^{+\infty}\left(\frac{\Lambda_T(\varepsilon)}{\Lambda_0}(S_T(\varepsilon)-X)\right)^\alpha dw^+(1-F(\varepsilon))-\\
-\lambda\int\limits_{S_T=-\infty}^X\left(\frac{\Lambda_T(\varepsilon)}{\Lambda_0}(X-S_T(\varepsilon))\right)^\beta dw^-(F(\varepsilon)).
\end{multline}
\end{document} 

